\chapter{Zaključak}
Cilj ovog rada bio je pronaći odgovarajuće rješenje za problem semantičke segmentacije ortopanograma, u slučaju kad na raspolaganju imamo relativno malo označenih, a relativno puno neoznačenih slika. U tu svrhu ukratko su predstavljeni modeli dubokog učenja, s naglaskom na duboke konvolucijske modele, te je dano par primjera takvih arhitektura na konkretnom problemu semantičke segmentacije. Zatim je predstavljen koncept generativnih modela, te su predstavljene njihove prednosti nad klasičnim diskriminativnim modelima. Odabrana je jedna konkretna generatorska arhitektura, cycleGAN, iz razloga što, uz označene, može iskoristiti i neoznačene slike prilikom treniranja. Arhitektura je trenirana da raspozna 2, 3 i 33 klase u slici ortopanograma. Za treniranje je na raspolaganju bilo 250 označenih slika za treniranje, 62 slike za validaciju, 61 za testiranje i slučajan podskup od 250 slika od ukupno 3950 neoznačenih slika. Rezultati modela su pokazali da, iako je cycleGAN bio uspješan u problemu segmentacije, nije pokazao značajno bolje rezultate u odnosu na najnaprednije diskriminativne modele (Deeplab), te je u slučaju 33 klase čak pokazao lošije rezultate. Konačno, dano je par prijedloga za moguća daljna istraživanja.