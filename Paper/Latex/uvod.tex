\chapter{Uvod}
U današnjem svijetu podataka iz dana u dan ima sve više. Međutim, iako je do podataka danas vrlo lako doći, izvući korisno znanje iz njih se može pokazati kao vrlo težak zadatak. Računalni vid je jedno od područja umjetne inteligencije koje na osnovu 2D slika i videozapisa - podataka, nastoji dobiti što više informacija, odnosno korisnog znanja. Primjene ovog područja su brojne, od autonomne vožnje i cestovne sigurnosti do primjena u zdravstvu i medicini. Jedna od metoda koja se zadnjih godina pokazala najuspješnijom su duboki neuronski modeli, matematički modeli učenja zasnovani na principu rada ljudskog mozga.

Međutim, da bi duboki modeli bili uspješni, potrebne su im velike količine označenih podataka, do kakvih je često vrlo teško doći. Kako bi se riješio taj problem, razvijena su brojna rješenja, među koje spada i polunadzirano učenje - prisutp učenju dubokih mreža gdje, uz dio označenih podataka, model koristi i određen broj neoznačenih podataka kako bi poboljšao učenje. Jedan od takvih modela je i cycleGAN, nadogradnja popularne GAN arhitekture, posebice korisne u primjenama u računalnom vidu, kad je broj označenih podataka ograničen. 

U ovom radu opisat ću osnove dubokih neuronskih mreža, s posebnim naglaskom na duboke konvolucijske modele. Predstavit ću i koncept generativnih mreža, te ću se posebno fokusirati na jednu konkretnu generativnu mrežu - cycleGAN. Predloženu arhitekturu primijenit ću za rješavanje problema semantičke segmentacije ortopanograma, što je problem koji ima brojne praktične primjene. Rezultate predložene arhitekture ću usporediti s rezultatima postojećih diskriminativnih mreža, te donijeti zaključak koji pristup je prikladniji za navedeni problem.